\documentclass[a4paper,11pt]{article}
\usepackage{amsmath,amssymb}
\usepackage{geometry}
\geometry{margin=2.5cm}

\title{Delegaten als universelle Wirkungsobjekte}
\author{}
\date{}

\begin{document}
\maketitle

\section*{1. Ausgangsdefinition}

Ein Delegat ist kein bloßes Sprachkonstrukt, sondern ein \emph{typisiertes, algebraisch-topologisches Wirkungsobjekt},  
das ausführbar, kombinierbar und hierarchisch einbettbar ist.

Er ist mehr als:

\begin{itemize}
    \item ein Funktionszeiger
    \item ein Lambda-Ausdruck
    \item ein Interface
    \item ein Callback
    \item eine Monade
\end{itemize}

Ein Delegat kann sein:

\begin{itemize}
    \item Funktionszeiger mit Typeninformationen, Generics und Templates
    \item Grundrechenarten (Addition, Subtraktion, Multiplikation, Potenz, Logarithmus, Wurzel)
    \item Winkel oder Richtungen
    \item Boolesche Mengenoperationen
    \item Operationen der Veränderung von Hierarchiestruktur und Koordinaten
\end{itemize}

\section*{2. Minimaldefinition eines Delegaten (Richtung)}

Im einfachsten Fall ist ein Delegat \emph{keine Funktion}, sondern lediglich eine Richtung.

\[
D \in V
\]

wobei \(V\) ein reeller Vektorraum ist. Beispiele:

\begin{itemize}
    \item Winkel \(\theta \in [0,2\pi)\)
    \item normierter Richtungsvektor \(v \in \mathbb{R}^n\)
    \item Richtung mit Intensität \(\alpha v\)
\end{itemize}

Linearkombination mehrerer Richtungen:

\[
D = \alpha \cdot D_1 + \beta \cdot D_2
\]

\section*{3. Hierarchie der Delegaten}

\textbf{Stufenmodell:}

\textbf{Stufe 0: Reine Richtung}
\[
D = \theta
\]

\textbf{Stufe 1: Gewichtete Richtung}
\[
D = \alpha \cdot \theta
\]

\textbf{Stufe 2: Linearkombination}
\[
D = \sum_i \alpha_i \cdot D_i
\]

\textbf{Stufe 3: Richtungsfeld}
\[
D(x) = \sum_i \alpha_i(x) \cdot D_i
\]

\textbf{Stufe 4: Ausführbarer Delegat}
\[
D(x) \mapsto \text{Wirkung}
\]

\section*{4. Algebraische Dimension (Grundrechenarten)}

\[
D_1 + D_2, \quad
D_1 - D_2, \quad
D_1 \cdot D_2, \quad
D^n, \quad
\log(D), \quad
\sqrt{D}
\]

Interpretation:

\begin{itemize}
    \item Addition: Überlagerung von Wirkungen
    \item Subtraktion: Maskierung oder Ausschluss
    \item Multiplikation: Verschränkung oder Kopplung
    \item Potenz: Intensivierung oder Iteration
    \item Logarithmus: Dämpfung oder Kompression
    \item Wurzel: Abschwächung oder Entfaltung
\end{itemize}

\section*{5. Objekt- und Typ-Hierarchie}

Delegaten können an Objekte gebunden sein und in Vererbungshierarchien existieren:

\[
\text{Delegate}<T_{\text{in}}, T_{\text{out}}>
\]

Beispiele:

\[
\text{LoggingDelegate}<T>, \quad
\text{SecurityDelegate}<T>, \quad
\text{ValidationDelegate}<T>
\]

Delegaten können vererbt, spezialisiert, polymorph verwendet und kombiniert werden.

\section*{6. Topologische Dimension}

Natürliche topologische Begriffe ergeben sich aus der Interpretation als Richtungen:

\begin{itemize}
    \item Nähe: kleiner Winkel zwischen Delegaten
    \item Opposition: Winkel nahe \(\pi\)
    \item Glättung: Mittelung mehrerer Richtungen
    \item Rand: Diskontinuitäten im Richtungsfeld
\end{itemize}

\[
\text{near}(D_1, D_2), \quad \text{smooth}(D)
\]

\section*{7. Erweiterte boolsche und hierarchische Dimension}

Ein Delegat kann auch boolesche Mengenoperationen oder Operationen der Veränderung von Hierarchiestruktur und Koordinaten ausführen:

\[
D := \langle T_{\text{in}} \to T_{\text{out}}, \mathcal{A}, \mathcal{T}, \mathcal{H}, \mathcal{B}, \mathcal{C} \rangle
\]

wobei:

\begin{itemize}
    \item \(\mathcal{A}\) Grundrechenarten
    \item \(\mathcal{T}\) Topologie
    \item \(\mathcal{H}\) Objekt- und Typ-Hierarchie
    \item \(\mathcal{B}\) boolsche Mengenoperationen (AND, OR, NOT)
    \item \(\mathcal{C}\) Operationen auf Hierarchien und Koordinaten
\end{itemize}

Beispiel:

\[
(D_1 \cup D_2) \cdot (\alpha \cdot D_3 + \beta D_4) \mapsto \text{veränderte Hierarchie oder Koordinaten}
\]

\section*{8. Algebraische Operationen zwischen Funktionszeigern}

Funktionszeiger können zueinander arithmetisch kombiniert werden, nicht nur auf Rückgabewerten, sondern auf den Delegaten selbst:

\[
D_1 \,\text{op}\, D_2, \quad \text{op} \in \{ +, -, \cdot, /, ^, \sqrt{}, \log \}
\]

\subsection*{8.1 Variablenzuordnung}

1. \textbf{Parallele Anwendung:} Jede Variable \(x_i\) wird direkt an den entsprechenden Delegaten übergeben:

\[
(D_1 + D_2)(x_1, x_2) := D_1(x_1) + D_2(x_2)
\]

2. \textbf{Kombinierte Anwendung:} Variablen werden auf eine beliebige Menge von Delegaten verteilt:

\[
(D_1 + D_2)(x) := D_1(f_1(x)) + D_2(f_2(x))
\]

mit \(f_1, f_2\) als Abbildungen der Eingangsdaten.

\[
(D_1 \cdot D_2 + \sqrt{D_3}) \circ (D_4 \cup D_5) \mapsto \text{komplexe Wirkungsstruktur}
\]

\section*{9. Zusammengefasste formale Definition}

\[
D := \langle T_{\text{in}} \to T_{\text{out}}, \mathcal{A}, \mathcal{T}, \mathcal{H}, \mathcal{B}, \mathcal{C} \rangle
\]

\section*{10. Schlussaussage}

\[
\boxed{\text{Delegaten sind kombinierbare, universelle Wirkungsobjekte: Funktionen, Richtungen, Logik und Hierarchie zugleich}}
\]

\end{document}

