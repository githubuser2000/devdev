\documentclass[a4paper,11pt]{article}
\usepackage{amsmath,amssymb}
\usepackage{geometry}
\usepackage{enumitem}
\geometry{margin=2.5cm}

\title{Delegierende Programmierung als höheres Paradigma}
\author{}
\date{}

\begin{document}
\maketitle

\section*{Ausgangsfrage}

Wenn es neben imperativer und deklarativer Programmierung
ein höheres, delegierendes Programmierparadigma mit Delegaten gäbe,
in dem man Funktionen beziehungsweise Delegaten addieren,
subtrahieren und multiplizieren kann, unter Einbeziehung von
Arithmetik und Topologie:

\begin{quote}
Zeige, dass dann manche Funktionsaufrufe mit Delegaten
noch einmal einfacher sind als mit Haskell als deklarativ-funktionaler
Programmiersprache und noch mehr als mit imperativer,
prozeduraler Programmierung.
\end{quote}

\section*{1. Vergleich der Paradigmen}

\subsection*{1.1 Imperativ / prozedural}

Imperative Programmierung beschreibt Programme primär als
Sequenzen von Befehlen mit explizitem Kontrollfluss:

\begin{itemize}
\item Reihenfolge ist semantisch wesentlich,
\item Zustand ist explizit,
\item Kombination von Verhalten erfolgt durch Kontrollstrukturen.
\end{itemize}

Funktionsaufrufe sind an Ablauf, Variablen und Bedingungen gebunden.
Die semantische Bedeutung ist über viele Codezeilen verteilt.

\subsection*{1.2 Deklarativ / funktional (z.\,B. Haskell)}

Funktionale Programmierung beschreibt Programme als Transformationen
von Werten:

\[
f \colon A \to B
\]

Kombination erfolgt durch:
\begin{itemize}
\item Funktionskomposition,
\item Applicatives,
\item Monaden.
\end{itemize}

Dies reduziert expliziten Zustand, führt jedoch zu:
\begin{itemize}
\item geschachtelten Abstraktionen,
\item komplexen Typkonstruktionen,
\item linear-sequentieller Semantik von Effekten.
\end{itemize}

\section*{2. Einführung des delegierenden Paradigmas}

Das delegierende Paradigma hebt die Abstraktionsebene erneut an.

Ein Delegat ist kein bloßer Funktionswert, sondern ein
\emph{Bedeutungs- oder Wirkungsobjekt}.

\[
D := \langle \text{Typ}, \text{Wirkung}, \text{Kontext} \rangle
\]

Delegaten sind:
\begin{itemize}
\item erstklassig,
\item algebraisch kombinierbar,
\item topologisch strukturiert,
\item ausführbar.
\end{itemize}

\section*{3. Algebraische Kombination von Delegaten}

Im delegierenden Paradigma existieren wohldefinierte Operationen:

\[
D_1 + D_2 \quad
D_1 - D_2 \quad
D_1 \cdot D_2
\]

Diese Operationen wirken nicht auf Rückgabewerte,
sondern auf die Bedeutung der Delegaten selbst.

Interpretation:
\begin{itemize}
\item Addition: Überlagerung von Wirkungen,
\item Subtraktion: Ausschluss oder Maskierung,
\item Multiplikation: Verschränkung oder Abhängigkeit.
\end{itemize}

Damit wird ein gesamter Verarbeitungsablauf zu einer einzigen
algebraischen Formel.

\section*{4. Vergleich an einem Funktionsaufruf}

Angenommen, ein Wert soll:
\begin{itemize}
\item validiert,
\item geloggt,
\item transformiert,
\item eingeschränkt
\end{itemize}
werden.

\subsection*{4.1 Imperativ}

Die Bedeutung verteilt sich auf:
\begin{itemize}
\item mehrere Kontrollstrukturen,
\item Bedingungen,
\item temporäre Variablen.
\end{itemize}

Entfernung oder Hinzufügung eines Aspekts erfordert strukturelle
Codeänderungen.

\subsection*{4.2 Funktional (Haskell)}

Die Bedeutung wird durch Monadenstapel oder Funktionskomposition
ausgedrückt:

\begin{itemize}
\item semantisch dichter als imperativ,
\item aber typologisch und syntaktisch komplex.
\end{itemize}

Kombination ist möglich, aber nicht algebraisch transparent.

\subsection*{4.3 Delegierend}

Der gesamte Ablauf wird als Delegatenkombination geschrieben:

\[
D = (V + L + T) \cdot R
\]

Der Aufruf reduziert sich auf:

\[
D(x)
\]

Hinzufügen oder Entfernen eines Aspekts geschieht lokal:

\[
D - L, \quad D + 2T
\]

Keine Kontrollstruktur, keine Monaden, keine Reihenfolgeannahmen
auf Quelltextebene.

\section*{5. Rolle der Topologie}

Delegaten sind nicht nur algebraische Objekte, sondern liegen
in einem Wirkungsraum mit topologischer Struktur.

\begin{itemize}
\item Nähe zwischen Delegaten entspricht ähnlicher Wirkung,
\item Glättung ersetzt harte Kontrollentscheidungen,
\item Randbedingungen ersetzen explizite Verzweigungen.
\end{itemize}

Damit wird Reihenfolge durch räumliche Struktur ersetzt.

\section*{6. Formale Überlegenheit}

Für jede imperative Prozedur $P$ existiert eine funktionale
Darstellung $F$ mit höherem Abstraktionsgrad.

Für jede funktionale Darstellung $F$ existiert eine delegierende
Darstellung $D$ mit geringerer kombinatorischer Komplexität:

\[
\text{AST-Tiefe}(D) < \text{AST-Tiefe}(F) < \text{AST-Tiefe}(P)
\]

Der Grund ist, dass Delegaten zusätzliche algebraische
und topologische Struktur tragen, die Funktionen nicht besitzen.

\section*{7. Schlussfolgerung}

Delegierende Programmierung arbeitet auf einer höheren Ebene
als imperative und deklarativ-funktionale Programmierung.

Sie kombiniert:
\begin{itemize}
\item Ausführbarkeit,
\item Algebra,
\item Topologie,
\item Bedeutung.
\end{itemize}

\[
\boxed{
\text{Manche Funktionsaufrufe sind delegierend strikter einfacher als funktional,
und funktional einfacher als imperativ}
}
\]

\end{document}

