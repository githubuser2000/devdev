\documentclass[11pt]{article}
\usepackage[utf8]{inputenc}
\usepackage[T1]{fontenc}
\usepackage{amsmath, amssymb, amsthm, mathtools}
\usepackage{geometry}
\geometry{margin=3cm}

\title{Typen perfektoider Strukturen}
\author{}
\date{}

\begin{document}
\maketitle

\section*{1. Perfektoide Körper (Charakteristik $0$)}

Ein nichtarchimedischer Körper $K$ (vollständig, topologischer Ring) heißt \emph{perfektoid}
für eine feste Primzahl $p$, wenn
\begin{enumerate}
  \item $K$ ist $p$-adisch vollständig,
  \item die Wertgruppe $|K^\times| \subset \mathbb{R}_{>0}$ ist $p$-divisibel,
  \item der Frobenius auf
  \[
    \mathcal{O}_K/p := \{x\in\mathcal{O}_K \mid |x|\le 1\}/p
  \]
  surjektiv ist:
  \[
    \varphi : \mathcal{O}_K/p \to \mathcal{O}_K/p,\quad x\mapsto x^p.
  \]
\end{enumerate}

Typisches Beispiel:
\[
  K
  = \widehat{\mathbb{Q}_p(p^{1/p^\infty})}
  = \widehat{\bigcup_{n\ge 0}\mathbb{Q}_p(p^{1/p^n})}.
\]
Auch $\mathbb{C}_p$ ist perfektoid.

\bigskip

\section*{2. Perfektoide Körper (Charakteristik $p$)}

Ein Körper $K$ in Charakteristik $p$ heißt perfektoid, wenn
\begin{itemize}
  \item $K$ vollständig und nichtarchimedisch ist,
  \item der Frobenius
  \[
    \varphi : \mathcal{O}_K \to \mathcal{O}_K,\quad x\mapsto x^p
  \]
  bijektiv ist.
\end{itemize}

Beispiel:
\[
  K = \mathbb{F}_p((t^{1/p^\infty}))
  := \widehat{\bigcup_{n\ge 0}\mathbb{F}_p((t^{1/p^n}))}.
\]

\bigskip

\section*{3. Tilt eines perfektoiden Körpers}

Zu jedem perfektoiden Körper $K$ (Charakteristik $0$) existiert der \emph{Tilt}
\[
  K^\flat := \varprojlim_{x\mapsto x^p} K
  = \{(x_0,x_1,\dotsc)\mid x_{n+1}^p = x_n\},
\]
mit komponentweiser Multiplikation und
\[
  (x_n)\oplus(y_n)
  :=
  \left(\lim_{m\to\infty}(x_{n+m}+y_{n+m})^{p^m}\right)_n.
\]

Dann gilt:
\[
  \mathrm{char}(K^\flat)=p, \qquad
  |(K^\flat)^\times| = |K^\times|,
\]
und $K^\flat$ ist wiederum perfektoid. Es besteht eine Äquivalenz
\[
  \text{Perfektoide Körper in char }0
  \;\longleftrightarrow\;
  \text{Perfektoide Körper in char }p,
\]
kurz:
\[
  K \;\leftrightsquigarrow\; K^\flat.
\]

\bigskip

\section*{4. Perfektoide Ringe}

Ein \emph{perfektoider Ring} $A$ ist ein $p$-adisch vollständiger, uniformer Banachring mit
\begin{enumerate}
  \item $p\in A$ ist topologisch nilpotent,
  \item der Frobenius
  \[
    \varphi : A/p \to A/p
  \]
  ist surjektiv,
  \item es existiert $\varpi\in A$ mit
  \[
    \varpi^p \mid p
    \quad\text{in }A,
    \qquad |\varpi|<1.
  \]
\end{enumerate}

Wichtige Unterklassen:
\begin{itemize}
  \item $p$-adisch vollständige perfektoide Ringe,
  \item perfektoide Tate-Ringe,
  \item torfreie vs.\ nicht-torfreie Varianten.
\end{itemize}

Typischer Ursprung:
\[
  A = \mathcal{O}_K,\quad
  K \text{ perfektoider Körper.}
\]

\bigskip

\section*{5. Strukturklassifikation}

\[
\boxed{
\begin{aligned}
\textbf{Perfektoide Strukturen}
&=
\begin{cases}
\text{Körper} \\
\qquad\begin{cases}
\mathrm{char}=0 \\
\mathrm{char}=p \\
\text{Tilt-Paare }(K,K^\flat)
\end{cases}\\[6pt]
\text{Ringe}\\
\qquad\begin{cases}
p\text{-adisch vollständige perfektoide Ringe}\\
\text{perfektoide Tate-Ringe}
\end{cases}
\end{cases}
\end{aligned}
}
\]

\end{document}

