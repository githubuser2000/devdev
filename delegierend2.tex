\documentclass[a4paper,11pt]{article}
\usepackage{amsmath,amssymb}
\usepackage{geometry}
\usepackage{enumitem}
\geometry{margin=2.5cm}

\title{Der Delegat als algebraisch-topologisches Wirkungsobjekt}
\author{}
\date{}

\begin{document}
\maketitle

\section*{Ausgangsfrage}

Ein Delegat kann ein Funktionszeiger mit Typeninformationen, Generics und Templates sein.  
Ein Delegat können auch die vier Grundrechenarten mit Potenz, Logarithmus und Wurzel sein.  
Als Drittes kann ein Delegat zu einem Objekt gehören, welches sich in einer Vererbungshierarchie befinden darf, weswegen Templates und Generics wichtig sind.  
Der Delegat darf als Viertes ``zu einer Arithmetik, zu einer Topologie, Topographie'', drei in einem, gehören.

\section*{1. Grundthese}

Ein Delegat ist kein bloßes Sprachkonstrukt, sondern ein
\emph{typisiertes, algebraisch-topologisches Wirkungsobjekt},
das ausführbar, kombinierbar und hierarchisch einbettbar ist.

Er ist damit mehr als:
\begin{itemize}
\item ein Funktionszeiger,
\item ein Lambda-Ausdruck,
\item ein Interface,
\item ein Callback,
\item eine Monade.
\end{itemize}

\section*{2. Erste Dimension: Delegat als typisierter Funktionszeiger}

Ein Delegat besitzt eine vollständige Typinformation:

\[
D \colon T_{\text{in}} \to T_{\text{out}}
\]

Er ist:
\begin{itemize}
\item erstklassig,
\item aufrufbar,
\item generisch parametrisierbar,
\item templatefähig.
\end{itemize}

Beispiele:
\[
D<int, float>, \quad D<T, T>, \quad D<User, Result>
\]

Diese Ebene entspricht der funktionalen Grundlage des Delegaten.

\section*{3. Zweite Dimension: Delegat als algebraisches Objekt}

Ein Delegat ist Element einer Algebra und erlaubt arithmetische Operationen
\emph{auf seiner Bedeutung}, nicht nur auf seinen Rückgabewerten.

\[
D_1 + D_2, \quad
D_1 - D_2, \quad
D_1 \cdot D_2, \quad
D^n, \quad
\log(D), \quad
\sqrt{D}
\]

Interpretation:
\begin{itemize}
\item Addition: Überlagerung von Wirkungen,
\item Subtraktion: Maskierung oder Ausschluss,
\item Multiplikation: Verschränkung oder Kopplung,
\item Potenz: Intensivierung oder Iteration,
\item Logarithmus: Dämpfung oder Kompression,
\item Wurzel: Abschwächung oder Entfaltung.
\end{itemize}

\section*{4. Dritte Dimension: Delegat als Objekt in einer Hierarchie}

Ein Delegat kann an ein Objekt gebunden sein und Teil einer
Vererbungshierarchie werden:

\[
\text{Delegate}<T_{\text{in}}, T_{\text{out}}>
\]

Beispiele:
\[
\text{LoggingDelegate}<T>, \quad
\text{SecurityDelegate}<T>, \quad
\text{ValidationDelegate}<T>
\]

Delegaten können:
\begin{itemize}
\item vererbt,
\item spezialisiert,
\item polymorph verwendet,
\item kombiniert
\end{itemize}
werden.

\section*{5. Vierte Dimension: Arithmetik, Topologie und Topographie}

Ein Delegat gehört gleichzeitig zu drei mathematischen Strukturebenen:

\subsection*{5.1 Arithmetik}
\begin{itemize}
\item Gewichtung,
\item Skalierung,
\item Kombination,
\item Intensität.
\end{itemize}

\subsection*{5.2 Topologie}
\begin{itemize}
\item Nähe zwischen Delegaten,
\item Offenheit und Geschlossenheit,
\item Randbedingungen,
\item Glättung von Übergängen.
\end{itemize}

Beispiel:
\[
\text{near}(D_1, D_2), \quad \text{smooth}(D)
\]

\subsection*{5.3 Topographie}
\begin{itemize}
\item Höhe: Priorität,
\item Steigung: Änderungsrate,
\item Becken: Einflussbereich,
\item Pfade: Wirkungsverläufe.
\end{itemize}

\section*{6. Zusammengefasste formale Definition}

\[
D := \langle T_{\text{in}} \to T_{\text{out}}, \mathcal{A}, \mathcal{T}, \mathcal{H} \rangle
\]

wobei:
\begin{itemize}
\item $\mathcal{A}$ eine algebraische Struktur,
\item $\mathcal{T}$ eine Topologie,
\item $\mathcal{H}$ eine Objekt- und Typ-Hierarchie
\end{itemize}
darstellen.

Ein kombinierter Delegat wird direkt aufgerufen durch:

\[
(D_1 + \alpha D_2 - D_3) \cdot R(x)
\]

\section*{7. Schlussaussage}

\[
\boxed{\text{Delegaten sind kombinierbare Bedeutungsfelder, nicht bloß Funktionen}}
\]

\end{document}

