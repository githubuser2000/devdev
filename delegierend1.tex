\documentclass[a4paper,11pt]{article}
\usepackage{amsmath,amssymb}
\usepackage{geometry}
\geometry{margin=2.5cm}

\title{Delegaten als Richtungen und Linearkombinationen}
\author{}
\date{}

\begin{document}
\maketitle

\section*{1. Minimaldefinition eines Delegaten}

Im einfachsten, nicht-trivialen Fall ist ein Delegat \emph{keine Funktion},
sondern lediglich eine Richtung.

\[
D \in V
\]

wobei $V$ ein reeller Vektorraum ist.  
Ein Delegat kann z.\,B. sein:
\begin{itemize}
\item ein Winkel $\theta \in [0,2\pi)$,
\item ein normierter Richtungsvektor $v \in \mathbb{R}^n$,
\item eine Richtung mit Intensität $\alpha v$.
\end{itemize}

In dieser Stufe besitzt der Delegat noch keine Ausführungssemantik,
sondern beschreibt lediglich eine \emph{Tendenz} oder \emph{Wirkungsrichtung}.

\section*{2. Kombination von Delegaten}

Das Zusammenrechnen von Delegaten ist im einfachsten Fall
\emph{keine Addition}, sondern eine \emph{Linearkombination}.

\[
D = \alpha \cdot D_1 + \beta \cdot D_2
\]

Dabei gelten:
\begin{itemize}
\item $\alpha, \beta \in \mathbb{R}$ als Gewichte,
\item $D_1, D_2 \in V$ als Richtungsdelegaten,
\item $D$ als resultierende Richtung.
\end{itemize}

Die gewöhnliche Addition ist der Spezialfall $\alpha = \beta = 1$.

\section*{3. Interpretation}

Die Linearkombination entspricht geometrisch der Überlagerung von Richtungen,
vergleichbar mit:
\begin{itemize}
\item Kräften in der Physik,
\item Gradientenfeldern,
\item Vektorfeldern.
\end{itemize}

Programmatische Bedeutung entsteht hier nicht durch Reihenfolge,
sondern durch Richtung, Gewicht und Überlagerung.

\section*{4. Hierarchie der Delegaten}

Es ergibt sich eine natürliche Skala:

\subsection*{Stufe 0: Reine Richtung}
\[
D = \theta
\]

\subsection*{Stufe 1: Gewichtete Richtung}
\[
D = \alpha \cdot \theta
\]

\subsection*{Stufe 2: Linearkombination}
\[
D = \sum_i \alpha_i \cdot D_i
\]

\subsection*{Stufe 3: Richtungsfeld}
\[
D(x) = \sum_i \alpha_i(x) \cdot D_i
\]

\subsection*{Stufe 4: Ausführbarer Delegat}
\[
D(x) \mapsto \text{Wirkung}
\]

Funktionen treten erst in höheren Stufen auf und sind
\emph{nicht grundlegend}.

\section*{5. Topologische Aspekte}

Durch die Interpretation als Richtungen ergeben sich natürliche
topologische Begriffe:

\begin{itemize}
\item Nähe: kleiner Winkel zwischen Delegaten,
\item Opposition: Winkel nahe $\pi$,
\item Glättung: Mittelung mehrerer Richtungen,
\item Rand: Diskontinuitäten im Richtungsfeld.
\end{itemize}

Topologie ist somit keine Zusatzstruktur,
sondern folgt direkt aus der geometrischen Natur der Delegaten.

\section*{6. Zusammenfassung}

Ein Delegat ist im einfachsten Fall lediglich eine Richtung.
Programmierung entsteht durch die Linearkombination vieler solcher Richtungen
und erst in späteren Stufen durch deren Ausführung.

\[
\boxed{\text{Delegaten sind Vektoren, nicht primär Funktionen}}
\]

\end{document}

